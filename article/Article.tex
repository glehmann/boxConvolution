%
% Complete documentation on the extended LaTeX markup used for Insight
% documentation is available in ``Documenting Insight'', which is part
% of the standard documentation for Insight.  It may be found online
% at:
%
%     http://www.itk.org/

\documentclass{InsightArticle}


%%%%%%%%%%%%%%%%%%%%%%%%%%%%%%%%%%%%%%%%%%%%%%%%%%%%%%%%%%%%%%%%%%
%
%  hyperref should be the last package to be loaded.
%
%%%%%%%%%%%%%%%%%%%%%%%%%%%%%%%%%%%%%%%%%%%%%%%%%%%%%%%%%%%%%%%%%%
\usepackage[dvips,
bookmarks,
bookmarksopen,
backref,
colorlinks,linkcolor={blue},citecolor={blue},urlcolor={blue},
]{hyperref}
% to be able to use options in graphics
\usepackage{graphicx}
% for pseudo code
\usepackage{listings}
% subfigures
\usepackage{subfigure}


%  This is a template for Papers to the Insight Journal. 
%  It is comparable to a technical report format.

% The title should be descriptive enough for people to be able to find
% the relevant document. 
\title{Efficient implementation of kernel filtering}

% Increment the release number whenever significant changes are made.
% The author and/or editor can define 'significant' however they like.
\release{0.00}

% At minimum, give your name and an email address.  You can include a
% snail-mail address if you like.
\author{Richard Beare{$^1$} {\small{and}} Ga\"etan Lehmann{$^2$}}
\authoraddress{{$^1$}Department of Medicine, Monash University, Australia.\\
{$^2$}INRA, UMR 1198; ENVA; CNRS, FRE 2857, Biologie du D\'eveloppement et
Reproduction, Jouy en Josas, F-78350, France}

\begin{document}
\maketitle

\ifhtml
\chapter*{Front Matter\label{front}}
\fi


\begin{abstract}
\noindent
% The abstract should be a paragraph or two long, and describe the
% scope of the document.
Kernel based filtering is one of the fundamental tools of image
analysis and processing. A number of approaches have been developed
that allow efficient implementation of such filters even when the
kernel size is large. This article reviews some of these methods and
provides introduces their ITK implementations.
\end{abstract}

\tableofcontents

\section{Introduction}
A kernel based filtering process replaces the pixel at the kernel
origin with the result of applying a function to all pixels defined by
the kernel. Many useful filters, including edge detection and gradient
filters, smoothing filters and rank and morphology filters fall into
this category. Direct implementations of such filters typically
involve visiting all pixels defined by the kernel in order to evaluate
the filter function. Such an approach is usually simple to implement
-- in ITK it is made simple by the neighborhood iterators -- but leads
to an algorithm complexity proportional to the number of pixels in the
kernel (or $O(n^d)$, where $n$ is the kernel size and $d$ is the
dimenionality). Such complexity tends to restrict application of such
filters to small kernels.

A number of classical methods exist for reducing this complexity to
more managable, in some cases kernel size independent, levels. ITK
already exploits such techniques for gaussian convolution
operations. This paper describes the classical techniques for
optimized mathematical morphology filters, rank filters and certain
convolution filters. These filters make the use of large kernels,
which can be very useful in many applications, practical on
conventional computing hardware.

\section{Separability and recursive implementations}
The two approaches most typically used to reduce complexity of kernel
based filters are separability and recursive computation, both of
which are used in the ITK implementation of gaussian convolution
filters. A separable filter implements a multidimensional kernel by
cascading several one dimensional kernels, therefore reducing
complexity from $O(n^d)$ to $O(nd)$. The second approach exploits
redundancy that might be present in the computations of kernel
functions at neighboring locations, leading, in some cases, to a complexity
independent of $n$.

\section{Mathematical morphology operations}
Two optimized forms of the mathematical morphology operations of
erosion, dilation, opening and closing are presented here. The kernel
is usually referred to as the structring element in mathematical
morphology. The methods described here are applied to ``flat''
structuring elements, that is structuring elements without
weights. The first method can be applied to arbitary structuring
elements while the second can be applied to line structuring elements.

\subsection{Arbitary structuring elements}
\label{sect:MMmovingHist}
The method described in \cite{Vandroogenbroeck96.3} relies on the
simple concept of an updatable histogram. A histogram is computed for
a kernel located at the first voxel. The histogram at the neighboring
voxel can then be computed by including newly included voxels and
removing newly excluded voxels. The list of included and excluded
voxels corresponding to movement in any direction can be computed when
the structuring element is created, and the direction with the
smallest number of changes should be selected as the direction for
sweeping the kernel across the image. The erosion or dilation at each
location is computed by selecting the minimum or maximum from the
histogram. This approach is very efficient when 8 or 16 bit pixels are
used because the histogram can be represented as an array, with place
holders tracking the current maximum or minimum increasing
performance. More sophisticated histogram representations are
necessary for larger pixel types. Our implementation uses c++ maps.

This methodology reduces the complexity from $O(n^d)$ to $O(n^{d-1})$,
while keeping the structuring element identical to the direct
implementation.

\subsection{Decomposition of structuring elements}
Morphological erosions and dilations are separable -- successive
dilations by orthogonal lines is equivalent to dilation by a rectangle
with sides equal to the line lengths. This means that any
hyper-rectangular structuring element can be constructed using lines
parallel to the image axes.

Approximations of more complex shapes, particulary circles and
ellipses can constructed using a number of lines at evenly spaced
angles \cite{Adams93}. It is difficult to create a structuring element
with a precisely defined radius using this method because line
structuring elements from which the circle structuring element is
composed must have odd length and there are practical limits due to
the realities of underlying digital grid representation of images. In
addition it is possible that the structuring elements may not be truly
translation invariant due to the representation of line used
(e.g. Bresenham) in the decomposition. However, precisely defined
radii are rarely critical when a large structuring element is called
for. An example of a structuring element created using line
structuring elements is shown in Figure \ref{fig:circledecomposition}. Composition of regular
shapes, such as hexagons and octagons is more accurate.

\begin{figure}[htbp]
\centering
\includegraphics{kernel}
\caption{Approximate circular structuring element, radius 25, constructed using 8 lines.\label{fig:circledecomposition}}
\end{figure}

It is also possible, in theory, to construct 3D structuring elements
in similar ways. The construction of a hyper-rectangle is trivial,
however construction of spheres is more problematic. The code
discussed later provides preliminary implementations based on some
platonic solids and various spherical approximateions, but further
testing and development is needed.

\subsection{Line structuring elements}
The decompositions discussed above are important because an efficient,
recursive, implementation of erosion and dilations along lines
exists. This method was introduced in \cite{Gil1993,vanHerk1992a} and
can compute an erosion or dilation in 3 operations per pixel,
independent of structuring element length. \cite{Gil2000} recently
reduced the cost to 1.5 pixels per pixel, but the procedure is more
complicated and there are reports of no speedup in reality.

The original algorithms utilize forward and backward running maxima
(for dilation) for the length of the structuring element from a pixel
of interest. The dilation can then be computed for a region the size
of the structuring element around the point of interest by comparing
values on the running extrema that are separated by the structuring
element length. This is illustrated in Figure \ref{fig:vHGWmethod}

\begin{figure}[htbp]
\centering
\includegraphics[scale=0.5]{vHGWexpl}
\caption{The van Herk, Gil, Werman method.\label{fig:vHGWmethod}}
\end{figure}

There has been a method published recently that offers improved
performance \cite{Vandroogenbroeck2005Morphological} and the ability
to perform a line opening directly (rather than using an
erosion/dilation cascade). This method has been implemented in the
filters discussed later in the article, but performance issues are not
yet clear. The method employs histograms and therefore needs more
complex data structures when applied to higher precision data,
potentially any speed advantage.

\section{Rank filters}
Efficient implemetations of median and rank filters can be carried out
using exactly the same approach as discussed for morphological
operations with arbitary structuring elements. The method was
originally proposed in \cite{Huang79}. The implementation discussed
later supports arbitary kernel shapes and pixel types.

Rank filters are not separable. However the performance benefits
offered by separability make it worth pretending they are. If median
filtering is being used to provide robust noise filtering or
background estimation then a separable approximation is worth
testing. This concept was originally proposed in \cite{Narendra81}.

\appendix



\bibliographystyle{plain}
\bibliography{InsightJournal,local}
\nocite{ITKSoftwareGuide}

\end{document}

