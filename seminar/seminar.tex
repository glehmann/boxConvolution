\documentclass[pdf,prettybox]{prosper}
\usepackage{graphicx}
\title{Efficient implementation of kernel filtering}
\author{Richard Beare{$^1$} {\small{and}} Ga\"etan Lehmann{$^2$}}
\institution{{$^1$}Department of Medicine, Monash University, Australia.\\
{$^2$}INRA, UMR 1198; ENVA; CNRS, FRE 2857, Biologie du D\'eveloppement et
Reproduction, Jouy en Josas, F-78350, France}

\Logo(1cm,-1cm){\includegraphics[height=0.5cm]{monash}\hspace{2cm}\includegraphics[height=0.5cm]{LogoINRA-Couleur}\hspace{2cm}\includegraphics[height=0.5cm]{mima2}}

\begin{document}
\maketitle

\begin{slide}{Introduction}
\begin{itemize}
\item Describe methods that reduce computational complexity of kernel filters by orders of magnitude.
\item Discuss the tradeoffs that need to be considered when using these methods.
\end{itemize}
\end{slide}

\begin{slide}{Motivation}
Large median filter $111 \times 111$ for background estimation in microscopy image.
\begin{itemize}
\item Direct implementation takes at least 20 minutes.
\item Efficient approximation takes less than 5 seconds.
\end{itemize}
\end{slide}

\overlays{5}{%
\begin{slide}{Improving filter performance}
\begin{itemstep}
\item Recursive computation
	\begin{itemize}
	\item An output value can be computed based on a neighbouring output.
	\item This reduces complexity by eliminating repeated computation.
	\end{itemize}
\item Separability
	\begin{itemize}
	\item Certain multidimensional kernels can be decomposed into a number of one dimensional operations.
	\end{itemize}
\end{itemstep}
\end{slide}
}

\begin{slide}{Improving filter performance}
\begin{itemize}
\item ITK Gaussian filtering classes use both approaches.
\end{itemize}
\end{slide}


\overlays{2}{%
\begin{slide}{Sliding window}
\onlySlide*{1}{%
\begin{centering}
\begin{figure*}[p]
\includegraphics[width=6cm]{sliding}
\end{figure*}
\end{centering}}%
\onlySlide*{2}{
\begin{centering}
\begin{figure*}[p]
\includegraphics[width=6cm]{slidingB}
\end{figure*}
\end{centering}}%
\end{slide}
}

\begin{slide}{Sliding window}
\begin{itemize}
\item Collect a summary of all pixels in the neighbourhood.
\item Compute the output value from the summary.
\item Move the window and update the summary.
\end{itemize}
\end{slide}

\begin{slide}{Sliding window}
Reduces complexity from $O(n^d)$ to $O(n^{d-1})$

The standard updatable summaries are:
\begin{itemize}
\item Sum or sum of squares for mean and variance computations.
\item Histograms for rank filters including mathematical morphology.
\end{itemize}
\end{slide}

\overlays{3}{%
\begin{slide}{Separability/Decomposition}
\begin{itemstep}
\item Some useful filters can be computed using a cascade of operations along lines.
\item Special forms of updateable computations exist for some operations along lines.
\item These lead to operations with complexity independent of line length.
\end{itemstep}
\end{slide}
}
\begin{slide}{Mathematical Morphology}
\begin{itemize}
\item van Herk/Gil Werman algorithm.
\item Erosions/dilations along lines with 3 operations per pixel.
\end{itemize}
\end{slide}

\begin{slide}{Mathematical Morphology}
\vspace{0.4cm}
\begin{centering}
\begin{figure*}
\includegraphics[width=7.5cm]{vHGWexpl}
\end{figure*}
\end{centering}
\end{slide}

\begin{slide}{Mathematical Morphology}
\begin{itemize}
\item Decomposition of shapes as follows:
	\begin{itemize}
	\item Hyper-boxes - cascade of line operations parallel to image axes.
	\item Appoximate circles in 2D.
	\item Experimental approximation to spheres.
	\end{itemize}
\end{itemize}
\end{slide}

\begin{slide}{Mathematical Morphology}
\vspace{0.4cm}
\begin{centering}
\begin{figure*}
\includegraphics[width=4cm]{kernel}
\end{figure*}
\end{centering}
\end{slide}

\begin{slide}{Robust smoothing filters}
\begin{itemize}
\item Median filters aren't separable.
\item Applying a cascade of line based rank filters produces a fast, robust smoother.
\end{itemize}
\end{slide}

\begin{slide}{Accumulation image methods}
\begin{itemize}
\item Useful for sums, means or variances of boxes.
\item One accumulation image can be used to evaluate lots of different kernel sizes.
\item Can also be used for binary morphology and binary median filters.
\end{itemize}
\end{slide}

\begin{slide}{Accumulation image methods}
\vspace{0.4cm}
\begin{centering}
\begin{figure*}
\includegraphics[width=7cm]{accum2D}
\end{figure*}
\end{centering}
\end{slide}

\begin{slide}{Accumulation image methods}
\vspace{0.4cm}
\begin{centering}
\begin{figure*}
\includegraphics[width=7cm]{accumMean2d}
\end{figure*}
\end{centering}
\end{slide}

\begin{slide}{Summary of filters}
\begin{itemize}
\item {\bf Mathematical Morphology using flat structuring elements} - arbitrary, box, polygon.
\item {\bf Accumulation image methods - mean and standard deviation}. Only one kernel size.
\item {\bf Rank filters} - arbitrary, boxes - separable approximation, also masked versions.
\end{itemize}
\end{slide}

\begin{slide}{Summary of complexities}
\begin{itemize}
\item Sliding window - $O(n^{d-1})$ - arbitary morphology, arbitary rank and mean filters.
\item Box convolution - constant complexity.
\item MM along lines - constant complexity - approximate circles, boxes.
\item Rank operations along lines - uses sliding windows and provides a fast, robust smoother.
\end{itemize}
\end{slide}


\end{document}
